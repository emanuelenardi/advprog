\documentclass[xcolor={dvipsnames, svgnames, x11names, table}, 10pt]{beamer}
\usepackage{./assets/preamble}

\title{Lezione di teoria}
\date{28 settembre 2021}
\institute{%
    \textbf{Obiettivi di apprendimento}:
    \begin{itemize}
        \item Ridefinizione degli operatori;
        \item ereditarietà.
    \end{itemize}%
}

% arara: xelatex: { synctex: no }
% arara: xelatex: { synctex: yes }
% arara: latexmk: { clean: partial }
\begin{document}

\frame{\titlepage}

\Sommario

\section{Esempio dei complessi}

\begin{frame}[fragile]{\texttt{complesso.h}}

\begin{columns}
    \column{\dimexpr\paperwidth-30pt}
    \only<1>{\cppfilepath[lastline=21]{complex/complesso.h}}
    \only<2>{\cppfilepath[firstline=23]{complex/complesso.h}}
\end{columns}
\vspace*{\fill}

\end{frame}

\begin{frame}[fragile]{\texttt{complesso.cpp}}

    \only<1>{\cppfilepath[firstline=1, lastline=17]{complex/complesso.cpp}}
    \only<2>{\cppfilepath[firstline=19, lastline=34]{complex/complesso.cpp}}
    \only<3>{\cppfilepath[firstline=36, lastline=52]{complex/complesso.cpp}}
    \only<4>{\cppfilepath[firstline=54, lastline=74]{complex/complesso.cpp}}

\end{frame}

\begin{frame}[fragile]{\texttt{main.cpp}}

\cppfilepath{complex/main.cpp}

\end{frame}

\begin{frame}{Breakout rooms: consegna}
    \begin{itemize}
        \item Scegliete nel vostro gruppo un operator dalla \href{https://en.wikipedia.org/wiki/Operators\_in\_C\_and\_C\%2B\%2B}{pagina di Wikipedia \enquote{Operators in \cplusplus} \ExternalLink} e discutetene la ridefinizione fra di voi
        \item Scrivete un esempio di classe con quell'operatore ridefinito. Un esempio \enquote{tecnico} non importa che la classe sia realistica.
        \item Non limitatevi a quelli semplici: si \enquote{vince} se nessun altro gruppo ha scelto lo stesso.
        \item Decidete chi lo presenta prima che vengano chiuse le \foreign{breakout rooms}.
    \end{itemize}
\end{frame}

\section{Ereditarietà in \cplusplus}

\begin{frame}{Ereditarietà in \cplusplus}
    \begin{itemize}
        \item Esistono tipi di ereditarietà: \texttt{public}, \texttt{private}, \texttt{protected};
        \item le stesse \foreign{keywords} della visibilità nella classe;
    \end{itemize}
    
    \begin{alertblock}{Attenzione}
        L'ereditarietà ha consequenza sulla visibilità.
    \end{alertblock}
\end{frame}

\begin{frame}{Visibilità delle classi in B}

\begin{center}
    \begin{tabular}{@{} *{5}{l} @{}}
    \toprule
        & & \multicolumn{3}{c}{Visibilità classe base \texttt{B}} \\
    \cmidrule{3-5}
        \multirow{5}{*}{\rotatebox[origin=c]{90}{\parbox[c]{3.5cm}{\centering Ereditarietà classe derivata \texttt{D}}}} &  & \texttt{public} & \texttt{private} & \texttt{protected} \\
    \cmidrule(l){2-5}
        & \texttt{public} & Everyone & B & B/D \\
    \cmidrule(l){2-5}
        & \texttt{public} & Everyone & B & B/D \\
    \cmidrule(l){2-5}
        & \texttt{private} & Everyone & \centered{-} & D \& \foreign{such} \\
    \cmidrule(l){2-5}
        & \texttt{protected} & D \& \foreign{such} & \centered{-} & D \& \foreign{such} \\
    \bottomrule
    \end{tabular}
\end{center}
    
\end{frame}

\begin{frame}{Visibilità delle classi nella classe base}

\begin{center}
    \begin{tabular}{@{} *{5}{l} @{}}
    \toprule
        & & \multicolumn{3}{c}{Visibilità classe base}\\
    \cmidrule(l){3-5}
        \multirow{5}{*}{\rotatebox[origin=c]{90}{\parbox[c]{3.5cm}{\centering Ereditarietà classe derivata}}} & & \texttt{public} & \texttt{private} & \texttt{protected} \\
    \cmidrule(l){3-5}
        & \texttt{public} & & & \\
    \cmidrule(l){2-5}
        & \texttt{public} & Public & Inaccessibile & Protected \\
    \cmidrule(l){2-5}
        & \texttt{private} & Private & Inaccessibile & Private \\
    \cmidrule(l){2-5}
        & \texttt{protected} & Protected & Inaccessibile & Protected \\
    \bottomrule
    \end{tabular}
\end{center}
    
\end{frame}

\begin{frame}{Istanza di una classe derivata}

Istanziata una classe derivata questa può essere acceduta sia come istanza della classe derivata sia come istanza della classe base.

In particolare i puntatori alla classe base possono essere usati per puntare ad un istanza della classe derivata.

\begin{alertblock}{Attenzione}
Un'istanza di una classe derivata può essere assegnata ad un'istanza di una classe base ma non il contrario.
\end{alertblock}

\end{frame}

\begin{frame}{Keyword \texttt{virtual}}

I metodi possono essere definiti \texttt{virtual}.

\'{E} il modo in cui in \cplusplus si dichiara il \foreign{late binding} rispetto all'\foreign{early binding}:
\begin{itemize}
    \item l'\textbf{\foreign{early binding}} viene risolto durante la compilazione (\foreign{compile-time});
    \item il \textbf{\foreign{late binding}} viene risolto a tempo di esecuzione (\foreign{run-time}).
\end{itemize}

\begin{block}{Quando si può usare il \foreign{late-binding}}
    \foreign{late binding} solo se si accede tramite puntatore o reference.
\end{block}

\end{frame}

\begin{frame}[t, fragile]{Esempio di \alert{classe base}: \texttt{a.h}}

    \only<1>{\cppfilepath[firstline=1, lastline=6]{inheritance/a.h}}
    \only<2>{\cppfilepath[firstline=8]{inheritance/a.h}}

\end{frame}

\begin{frame}[fragile]{Esempio di \alert{classe base}: \texttt{a.cpp}}

    \only<1>{\cppfilepath[lastline=20]{inheritance/a.cpp}}
    \only<2>{\cppfilepath[firstline=22]{inheritance/a.cpp}}

\end{frame}

\begin{frame}[fragile]{Esempio di \alert{classe derivata}: \texttt{b.h}}

    \only<1>{\cppfilepath{inheritance/b.h}}

\end{frame}

\begin{frame}[fragile]{Esempio di \alert{classe derivata}: \texttt{b.cpp}}

    \only<1>{\cppfilepath[lastline=17]{inheritance/b.cpp}}
    \only<2>{\cppfilepath[firstline=19]{inheritance/b.cpp}}

\end{frame}

\begin{frame}{Esempio di classe \alert{\texttt{main}}: \texttt{main.cpp}}
    
    \only<1>{\cppfilepath[lastline=7]{inheritance/main.cpp}}
    \only<2>{\cppfilepath[firstline=9, lastline=27]{inheritance/main.cpp}}
    \only<3>{\cppfilepath[firstline=29]{inheritance/main.cpp}}
    
\end{frame}

\section{Classe puramente virtuale}

\begin{frame}[fragile]{Classe puramente virtuale}

\begin{block}{Cos'è?}
    Una classe puramente virtuale ha almeno un metodo puramente virtuale, ovvero che non viene implementato.
\end{block}

La dichiarazione di un metodo virtuale nella classe è seguita da \enquote{\texttt{=0;}}

\begin{minted}{cpp}
class A {
    [...]
    virtual int metodo() = 0;
    [...]
};
\end{minted}
    
\begin{alertblock}{Attenzione}
Una classe puramente virtuale \textbf{non} ha istanze.
\end{alertblock}

\end{frame}

\begin{frame}{Esempio di classi puramente virtuali}

\begin{itemize}
    \item Classi puramente virtuali: personaggio e strumento;
    \item Classi derivate specifiche.
\end{itemize}

\end{frame}

\Riconoscimenti

\end{document}