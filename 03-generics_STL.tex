\documentclass[xcolor={dvipsnames, svgnames, x11names, table}, 10pt]{beamer}
\usepackage{./assets/preamble}

\title{Programmazione generica e \texttt{STL}}
\date{5 ottobre 2021}
\institute{%
    \textbf{Obiettivi di apprendimento}:
    \begin{itemize}
        \item Cos'è la programmazione generica;
        \item Quali sono gli operatori di conversione di tipo e come usarli;
        \item Cosa contiene la Standard Template Library.
    \end{itemize}%
}

% \includeonlyframes{current}

% arara: xelatex: { synctex: no }
% arara: xelatex: { synctex: yes }
% arara: latexmk: { clean: partial }
\begin{document}

\frame{\titlepage}

\section*{Riassunto lezione precedente}

\begin{frame}{Argomenti trattati nell'ultima lezione}
\begin{itemize}
    \item Come si effettua l'\foreign{overloading} degli operatori;
    \item Ereditarietà in \cplusplus;
    \item Cosa sono e come si definizione le classi puramente virtuali.
\end{itemize}

\end{frame}

\Sommario

\section{Programmazione generica}

\subsection*{Che cos'è la programmazione generica}

\begin{frame}{Programmazione generica}

    \alert<1>{Si intende la possibilità data da un linguaggio di \textbf{rappresentare tipi e implementare algoritmi che abbiano un tipo come parametro}.}

    \pause

    \alert<2>{Il tipo parametrico viene poi specificato:
    \begin{itemize}
        \item a tempo di \textbf{compilazione} (es.\ \texttt{templates} in \cplusplus);
        \item a tempo di \textbf{esecuzione} (es. \texttt{generics} in \textsf{\textbf{Java}}).
    \end{itemize}}
    
    \pause
    
    \alert<3>{\begin{block}{Qual è lo scopo?}
    Implementare algoritmi che siano \textbf{indipendenti dal tipo su cui operano} (\enquote{generici} appunto).
    \end{block}}
    
    \pause
    
    \alert<4>{Idealmente al \textbf{massimo livello di astrazione possibile}.
    Prendiamo come esempio lo \foreign{swapping} di variabili.}

\end{frame}

\subsection*{Esempio di scambio di variabili}

\begin{frame}[t, fragile]

    \frametitle<1>{Swap overloading [1/3] --- \alert{definizione di funzioni}}
    \only<1>{\cppfilepath[lastline=18]{example/example_overloading.cpp}}
    
    \frametitle<2>{Swap overloading [2/3] --- \alert{scambio variabili stringhe e intere}}
    \only<2>{\addpage\cppfilepath[firstline=20, lastline=39]{example/example_overloading.cpp}}
    
    \frametitle<3>{Swap overloading [3/3] --- \alert{scambio variabili decimali}}
    \only<3>{\addpage
        \cppfilepath[firstline=41]{example/example_overloading.cpp}
        
        \pause
        \begin{alertblock}{Attenzione}
        Non è possibile effettuare la conversione automatica fra \texttt{double} e \texttt{\&int}.
        \end{alertblock}%
    }

\end{frame}

% \begin{comment}
\subsection{Conversione di tipo (\texorpdfstring{\foreign{type conversion}}{type conversion})}

\begin{frame}[t]{Le conversioni di tipo [1/3]}

\alert<1>{\textbf{\foreign{type conversion}}, \textbf{\foreign{type casting}}, \textbf{\foreign{type coercion}}, e \textbf{\foreign{type juggling}} sono modi diversi di cambiare un'espressione da un tipo di dati a un altro.}
\pause
\alert<2>{Un esempio potrebbe essere la conversione di un valore intero in un valore in virgola mobile o la sua rappresentazione testuale come una stringa, e viceversa.}
\pause
\alert<3>{Le conversioni di tipo \textbf{possono sfruttare} alcune caratteristiche delle gerarchie dei tipi o delle rappresentazioni dei dati.}
\pause

\alert<4>{Due aspetti importanti di una conversione di tipo sono se avviene \textbf{implicitamente} (automaticamente) o \textbf{esplicitamente}}, o \alert<5>{se \textbf{la rappresentazione dei dati sottostante viene convertita da una rappresentazione in un'altra}}, o \alert<6>{\textbf{se una data rappresentazione è semplicemente reinterpretata}} come la rappresentazione di un altro tipo di dati.
% In generale, possono essere convertiti sia i tipi di dati primitivi che composti.

\end{frame}

\begin{frame}[t]{Le conversioni di tipo [2/3]}

Ogni linguaggio di programmazione ha le sue regole su come i tipi possono essere convertiti: \alert<1>{i linguaggi con \textbf{tipizzazione forte} (\foreign{strong typing}) in genere \textbf{eseguono poche conversioni implicite} e \textbf{scoraggiano la reinterpretazione delle rappresentazioni}}, mentre \alert<2>{i linguaggi con \textbf{tipizzazione debole} (\foreign{weak typing}) eseguono molte conversioni implicite tra i tipi di dati.}

\uncover<3>{\alert<3>{Un linguaggio con tipizzazione debole spesso \textbf{consente di forzare il compilatore a interpretare arbitrariamente un elemento con rappresentazioni diverse}}}.%, questo può essere un errore di programmazione non ovvio o un metodo tecnico per gestire direttamente l'hardware sottostante.

\end{frame}

\begin{frame}[t]{Le conversioni di tipo [3/3]}

\alert<1>{Nella maggior parte delle lingue, la parola \enquote{coercizione} (\textbf{\foreign{coercion}}) \textbf{viene utilizzata per denotare una conversione implicita}, sia durante la compilazione che durante il \foreign{runtime}.}
\pause

\alert<2>{Ad esempio, in un'espressione che mescola numeri interi e in virgola mobile (come \texttt{5 + 0.1}), il compilatore convertirà automaticamente la rappresentazione intera in rappresentazione in virgola mobile in modo che le frazioni non vengano perse.}
\pause

\alert<3>{\textbf{Le conversioni di tipo esplicite sono indicate scrivendo codice aggiuntivo} (ad esempio aggiungendo identificatori di tipo o chiamando funzioni integrate) o scrivendo funzioni di conversione per il compilatore da utilizzare quando, in assenza delle stesse, il compilatore interromperebbe l'esecuzione con una mancata corrispondenza del tipo (\foreign{type mismatch}).}

% https://en.wikipedia.org/wiki/Type_conversion

\end{frame}

\subsubsection{Conversione di tipo nei linguaggi \texttt{C}}

\begin{frame}[t, fragile]{La conversione di tipo nei linguaggi \texttt{C}}

\uncover<1->{Nella famiglia dei linguaggi C, la parola \enquote{\foreign{cast}} si riferisce tipicamente ad una \textbf{conversione di tipo esplicita}.} %(al contrario di una conversione implicita), causando qualche ambiguità sul fatto che si tratti di una reinterpretazione di un \foreign{bit-pattern} o di una conversione di rappresentazione di dati reali.}
% More important is the multitude of ways and rules that apply to what data type (or class) is located by a pointer and how a pointer may be adjusted by the compiler in cases like object (class) inheritance.

\uncover<2->{\cplusplus è un \textbf{linguaggio fortemente tipato}.
\textbf{Molte conversioni}, soprattutto quelle che implicano una diversa interpretazione del valore, \textbf{richiedono una conversione esplicita}.}
\uncover<2->{\textbf{Finora conosciamo due notazioni}: quella funzionale e quella simile a C.}
\begin{overprint}
    
    \onslide<3->
    \begin{minted}{cpp}
    short a = 2000;
    int b;
    
    b = (int) a; // c-like cast notation
    b = int (a); // functional notation
    \end{minted}

\end{overprint}
\only<4>{\textbf{Quali sono} gli operatori di conversione di tipo che ci mette a disposizione il linguaggio? \textbf{Come si differenziano} fra di loro? \textbf{Quando usare} un tipo di conversione al posto di un altro?}

% https://www.cplusplus.com/doc/oldtutorial/typecasting/
% https://www.cplusplus.com/doc/tutorial/typecasting/
\end{frame}

\subsubsection{Tipi di casting in \cplusplus}

\begin{frame}[t, fragile]{Tipi di casting in \cplusplus}

    Esistono cinque diverse \alert<1->{\textbf{conversioni di tipo}} in \cplusplus:

    \begin{enumerate}[<+- | alert@+>]
        \item \textbf{implicita} (\foreign{using the assignment operator});
        \item \textbf{esplicita} (\foreign{using the cast notation});
        \item \textbf{statica} (\foreign{using} \texttt{static\_cast});
        \item \textbf{dinamica} (\foreign{using} \texttt{dynamic\_cast});
        \item \textbf{costante} (\foreign{using} \texttt{const\_cast});
        \item \textbf{reinterpretata} (\foreign{using} \texttt{reinterpret\_cast}).
    \end{enumerate}
    
    \begin{overprint}
    \onslide<1>
    \begin{minted}{cpp}
    int a = 5, b = 2;
    int c = a/b; // 5/2 = 2.5 -> float
    \end{minted}
    
    \begin{block}{Quando utilizzarlo?}
    La conversione di tipo implicita\dots
    \end{block}
    
    \onslide<2>
    \begin{minted}{cpp}
    short a = 2000;
    int b;
    
    b = (int) a; // c-like cast notation
    b = int (a); // functional notation
    \end{minted}
        
    \begin{block}{Quando utilizzarlo?}
    La conversione di tipo esplicita\dots
    \end{block}
    
    \onslide<3>
    \begin{minted}{cpp}
    int x=8, y=3;
    float divisione;
    divisione = static_cast<float>(x)/y;
    \end{minted}
    
    \begin{block}{Quando utilizzarlo?}
    Si utilizza quando si è sicuri della validità della conversione perchè non c'è nessun controllo sulla compatibilità di tipi.
    \end{block}
    
    \onslide<4>
    \begin{minted}{cpp}
    4
    \end{minted}
        
    \begin{block}{Quando utilizzarlo?}
    La conversione di tipo dinamica\dots
    \end{block}
    
    \onslide<5>
    \begin{minted}{cpp}
    5
    \end{minted}
        
    \begin{block}{Quando utilizzarlo?}
    La conversione di tipo costante\dots
    \end{block}
    
    \onslide<6>
    \begin{minted}{cpp}
    class A { /* ... */ };
    class B { /* ... */ };
    A *a = new A;
    B *b = reinterpret_cast<B*>(a);
    \end{minted}
    
    \begin{block}{Quando utilizzarlo?}
    La conversione di tipo reinterpretata\dots
    \end{block}
    \end{overprint}
    
    % https://stackoverflow.com/questions/28235805/type-casting-in-c
    % https://stackoverflow.com/questions/332030/when-should-static-cast-dynamic-cast-const-cast-and-reinterpret-cast-be-used
\end{frame}
% \end{comment}

\subsection{Esempi con codice}

\subsubsection{Swap \texttt{void*} C-style casting}

\begin{frame}[t, fragile]

    \frametitle<1>{Swap \texttt{void*} C-style casting [1/3]}
    \only<1>{\cppfilepath[lastline=10]{example/example_void_pointer1.cpp}}
    
    \frametitle<2>{Swap \texttt{void*} C-style casting [2/3]}
    \only<2>{\addpage\cppfilepath[firstline=12]{example/example_void_pointer1.cpp}}

\end{frame}

\subsubsection{Swap \texttt{void*} \texttt{static\_casting}}

\begin{frame}[t, fragile]

\begin{columns}
    \column{\dimexpr\paperwidth-30pt}
    
    \frametitle<1>{Swap \texttt{void*} \alert{\texttt{static\_casting}} [1/2]}
    \only<1>{\cppfilepath[lastline=10]{example/example_void_pointer2.cpp}}
    
    \frametitle<2>{Swap \texttt{void*} \alert{\texttt{static\_casting}} [2/2]}
    \only<2>{\addpage\cppfilepath[firstline=12]{example/example_void_pointer2.cpp}}
    
    % \frametitle<3>{Swap \texttt{void*} \alert{\texttt{static\_casting}} [3/3]}
    % \only<3>{\addpage\cppfilepath[firstline=20]{example/example_void_pointer2.cpp}}

\end{columns}

\end{frame}

\subsection{Swap template}

\begin{frame}[t, fragile]

\frametitle<1>{Swap template --- \alert{main} [1/2]}
\only<1>{\cppfilepath[lastline=11]{example/example_template.cpp}}

\frametitle<2>{Swap template --- \alert{main} [2/2]}
\only<2>{\addpage\cppfilepath[firstline=13]{example/example_template.cpp}}

\end{frame}

\begin{frame}[t, fragile]{Swap template --- \alert{A.h} e \alert{B.h}}

\cppfilepath{example/virtual/A.h}
\cppfilepath{example/virtual/B.h}

\end{frame}

\begin{frame}[t, fragile]{Swap template --- \alert{B.cpp}}

\cppfilepath{example/virtual/B.cpp}

\end{frame}

\begin{frame}[t, fragile]

\cppfilepath{example/virtual/example_hierarchy.cpp}

\end{frame}

\subsubsection*{Diversi tipi}

\colorlet{type1}{lime}
\colorlet{type2}{Yellow}
\colorlet{type3}{YellowOrange}
\setlength{\fboxsep}{2pt}

\begin{frame}[c, fragile]{Diversi tipi}
\begin{minted}[linenos=false, fontsize=\large, baselinestretch=0.2, escapeinside=||]{cpp}
template |\colorbox{type1}{<typename T>}|
|\colorbox{type1}{T}| min (|\colorbox{type1}{T}| a, |\colorbox{type1}{T}| b) {
    return a < b ? a : b;
}

template |\colorbox{type2}{<typename T1>}|, |\colorbox{type3}{typename T2>}|
|\colorbox{type2}{T1}| min (|\colorbox{type2}{T1}| a, |\colorbox{type3}{T2}| b) {
    return a < b ? a : b;
}

\end{minted}

\end{frame}

\subsubsection*{Class Pair}

\begin{frame}[t, fragile]{Class Pair}

\begin{columns}[t]

    \column{0.46\textwidth}
    \begin{minted}{cpp}
    template <typename F, typename S>
    class Pair {
    public:
        Pair(const F& f, const S& s);
        F get_first() const;
        S get_second() const;
    private:
        F first;
        S second;
    };
    \end{minted}
    
    \column{0.57\textwidth}
    \begin{minted}[firstnumber=12]{cpp}
    template <typename F, typename S>
    Pair<F,S>::Pair(const F& f, const S& s) {
        first = f;
        second = s;
    };
    
    template <typename F, typename S>
    F Pair<F,S>::get_first() const {
        return first;
    };
    
    
    template <typename F, typename S>
    S Pair<F,S>::get_second() const {
        return second;
    };
    \end{minted}

\end{columns}
\end{frame}

\section{Standard Template Library}

\begin{frame}[t]{Standard Template Library}

    La \textbf{Standard Template Library} (abbraviata con STL) è una libreria software inclusa nella libreria standard del linguaggio \cplusplus e fornisce quattro componenti:
    \begin{enumerate}[<+- | alert@+>]
        \item algoritmi (\foreign{algorithms});
        \item contenitori (\foreign{containers});
        \item funzioni (\foreign{functions});
        \item iteratori (\foreign{iterators}) una \textbf{generalizzazione dei puntatori}.
    \end{enumerate}

    \begin{center}
        \begin{tabular}{@{} ccc @{}}
            \cmidrule(lr){1-2}
        \end{tabular}
    \end{center}

    \only<1>{
    Nella STL \textbf{sono inclusi numerosi algoritmi} per eseguire operazioni come la \textbf{ricerca} e l'\textbf{ordinamento}.
    Tali algoritmi sono comunemente utilizzati per la manipolazione dei \foreign{container} in maniera indiretta, cioè solo tramite iteratori.
    Molti di questi algoritmi operano su un intervallo del container definito dall'utente tramite due iteratori che indicano gli estremi dell'intervallo.}

    \only<2>{
    I contenitori della STL si dividono in \textbf{sequenziali} e \textbf{associativi}.
    A loro volta, una parte dei contenitori sequenziali può essere definita come \textbf{adattatori}, in quanto sono in effetti delle interfacce ridotte e specializzate dei contenitori principali che non implementano iteratori nella loro interfaccia.}

\end{frame}

\subsection{Containers}

\newlength\origleftmargini
\setlength\origleftmargini\leftmargini
\setlength{\leftmargini}{10pt}
\begin{frame}{Containers}

\begin{columns}[T,onlytextwidth]
\column{0.35\textwidth}
\textbf{Sequence} containers:
\begin{itemize}
    \item \texttt{vector}
    \item \texttt{list}
    \item \texttt{set}
    \item \texttt{map}
\end{itemize}

\column{0.35\textwidth}
Container \textbf{adaptors}:
\begin{itemize}
    \item \texttt{stack}
    \item \texttt{queue}
    \item \texttt{priority\_queue}
\end{itemize}

\column{0.35\textwidth}
\textbf{Associative} containers:
\begin{itemize}
    \item \texttt{set}
    \item \texttt{multiset}
    \item \texttt{priority\_queue}
    \item \texttt{map}
    \item \texttt{multimap}
\end{itemize}

\end{columns}

\vspace{10pt}
\textbf{Unordered associative} containers:
\begin{columns}[T,onlytextwidth]
\column{0.4\textwidth}
\begin{itemize}
    \item \texttt{unordered\_set}
    \item \texttt{unordered\_multiset}
\end{itemize}

\column{0.58\textwidth}
\begin{itemize}
    \item \texttt{unordered\_map}
    \item \texttt{unordered\_multimap}
\end{itemize}
\end{columns}

\vspace*{\fill}

\href{https://www.cplusplus.com/reference/stl/}{Approfondisci su \texttt{cplusplus.com} \ExternalLink}

\end{frame}
\setlength{\leftmargini}{\origleftmargini}

\subsection{Iterators}

\begin{frame}[t]{Iterators}

Un \texttt{iterator} è un \textbf{oggetto che, puntando a un elemento in un intervallo di elementi} (come un array o un contenitore), ha la capacità di scorrere (\foreign{to iterate}) gli elementi di quell'intervallo utilizzando un insieme di operatori (con almeno l'incremento \enquote{\texttt{+}\(\!\)\texttt{+}} e l'operatore di dereferenziazione \enquote{\texttt{*}}).

\pause

\textbf{La forma più ovvia di iteratore è un puntatore}: un puntatore può puntare agli elementi di un array e scorrere attraverso essi utilizzando l'operatore di incremento \enquote{\texttt{+}\(\!\)\texttt{+}}, ma sono possibili altri tipi di iteratori.
Ad esempio, ogni tipo di contenitore (come una \texttt{list}) ha un tipo specifico di iteratore progettato per scorrere i suoi elementi.

\pause

\begin{alertblock}{Attenzione}
Nota che mentre un puntatore è una forma di iteratore, non tutti gli iteratori hanno la stessa funzionalità dei puntatori.
\end{alertblock}

\href{https://www.cplusplus.com/reference/iterator/}{Approfondisci su \texttt{cplusplus.com} \ExternalLink}

\end{frame}

\begin{frame}{}

\includegraphics[width=\textwidth]{}

\end{frame}

\subsection{Algorithms}

\begin{frame}[t]{Algorithms}

L'\foreign{header} \texttt{\(\textbf{<}\)algorithm\(\,\textbf{>}\)} \alert<1>{\textbf{definisce una raccolta di funzioni}} appositamente progettate per essere utilizzate su intervalli di elementi (\foreign{range}).

\pause

Un \foreign{range} è \alert<2>{\textbf{una qualsiasi sequenza di oggetti}} alla quale è possibile accedere tramite iteratori o puntatori, come un \texttt{array} o un'istanza di alcuni contenitori STL.
\pause
Si noti, tuttavia, che \alert<3>{\textbf{gli algoritmi operano tramite iteratori direttamente sui valori}}, \alert<3>{\textbf{non influenzando in alcun modo la struttura di ogni possibile contenitore}} (non influisce mai sulla dimensione o sull'allocazione dello spazio di archiviazione del contenitore).

\pause
Quindi gli algoritmi sono:
\begin{itemize}
    \item \textbf{basati sugli \foreign{iterators}};
    \item \textbf{generici anche rispetto ai \foreign{containers}}.
\end{itemize}

\vspace*{\fill}

\href{https://www.cplusplus.com/reference/algorithm/}{Approfondisci su \texttt{cplusplus.com} \ExternalLink}

\end{frame}

\subsection{Esempio di utilizzo di \texttt{STL}}

\begin{frame}[t, fragile]{Esempio di utilizzo di STL}

    \frametitle<1>{a.h}
    \only<1>{\cppfilepath{STL/a.h}}
    
    \frametitle<2>{a.cpp}
    \only<2>{\addpage\cppfilepath{STL/a.cpp}}
    
    \frametitle<3>{b.h}
    \only<3>{\addpage\cppfilepath{STL/b.h}}
    
    \frametitle<4>{b.cpp}
    \only<4>{\addpage\cppfilepath{STL/b.cpp}}
    
    \frametitle<5>{main.cpp [1/2]}
    \only<5>{\addpage
    \begin{columns}[t]
    
        \column{0.35\textwidth}
        \cppfilepath[lastline=15]{STL/main.cpp}
        
        \column{0.65\textwidth}
        \cppfilepath[firstline=17, lastline=31]{STL/main.cpp}

    \end{columns}
    }
    
    \frametitle<6>{main.cpp [2/2]}
    \only<6>{\addpage\cppfilepath[firstline=33]{STL/main.cpp}}

\end{frame}

%\Riconoscimenti

\end{document}